%
% Copyright (C) 2019 Javier Bezos http://www.texnia.com
%
% This file may be distributed and/or modified under the conditions of
% the MIT License. A version can be found at the end of this file.
%
% Repository: https://github.com/jbezos/esindex
%

\documentclass[a4paper]{ltxguide}

\title{\textsf{gindex}\\\large Version 0.1}

\author{Javier Bezos\\\texttt{http://www.texnia.com}}

\date{2019-10-01}

\raggedright
\parskip=.8ex
\advance\oddsidemargin-.7cm
\advance\textwidth2cm
\addtolength{\textheight}{3.5cm}
\addtolength{\topmargin}{-2cm}

\usepackage{xcolor,bera}

\definecolor{notes}{rgb}{.75, .3, .3}%

\makeatletter
\def\@begintheorem#1#2{%
  \list{}{}%
  \global\advance\@listdepth\m@ne
  \item[{\sffamily\bfseries\color{notes}\MakeUppercase{#1}}]}%
\makeatother
\newtheorem{warning}{Warning}
\newtheorem{note}{Note}
\newtheorem{example}{Example}

\begin{document}

\vspace*{1cm}
{\fontsize{48}{48}\selectfont \color{notes}{gindex}\par}
{\LARGE Formatting indexes with \LaTeX\par}
\vspace*{1ex}
Version 0.1 (2019-10-01)\par
Javier Bezos (\texttt{http://www.texnia.com})

\vspace*{6ex}

This package works in conjunction with a \textit{\bfseries g}eneric
\verb|ist| file, which provides a way to \textit{\bfseries g}enerate
the format of index entries from within \LaTeX. With standard classes,
it works out of the box with some predefined values, which you can (and
in fact should) redefine with |\renewcommand|.

Note it is version 0.1. Changes of the existing features are unlikely
to happen (but who knows), but new features, on the other hand, will be
added in the near future.

What it does is to generate blocks like this for each first-level entry:
\begin{verbatim}
  \setcounter{indexsubitems}{2}%     The number of subitems
  \indexitem{blah1}{1}%
    \indexsubitem{subblah1}{3}%
    \indexsubitem{subblah2}{5}%
  \indexnoitem{}{}%                  Usually dummy
\end{verbatim}

As you can see, when |\indexitem| is executed, you know the number of
subitems, which is necessary for some styles.

The default values are:
\begin{verbatim}
\newcommand\indexitem[2]{\item #1\ifx\\#2\\\else, #2\fi}
\newcommand\indexsubitem[2]{\subitem #1\ifx\\#2\\\else, #2\fi}
\newcommand\indexsubsubitem[2]{\subsubitem #1\ifx\\#2\\\else, #2\fi}
\end{verbatim}
For example, to replace the comma by a quad:
\begin{verbatim}
\renewcommand\indexitem[2]{\item #1\ifx\\#2\\\else\quad #2\fi}
\end{verbatim}

It is guaranteed that these command can take up to 5 arguments. If you
define |\indexitem| with all of them, the parameters in the previous
example will be:
\begin{enumerate}
  \item |blah1|
  \item |1|
  \item |\indexsubitem|
  \item |subblah1|
  \item |3|
\end{enumerate}
This explains the line |\indexnoitem{}{}| above, which by default does
nothing. By inspecting the third parameter you decide how to format the
current parameter (of course, in most cases you must emit it again at
the end of your definition of |\indexitem|).

Subsubitems are allowed, but not counted. Only subitems are counted.


There is a tool to define the most usual entry format, which is used in
the following way:
\begin{verbatim}
\renewcommand\indexitem[2]{%
  \indexitemhang{0pt}%       First line left margin
                {10pt}%      Second, third...  lines left margin
                {}%          After entry if no locator
                {: }%        Entry-locator separator
                {#1}%        Entry
                {#2}}%       Locator
\end{verbatim}

Other macros are:
\begin{itemize}
\item \verb|\indexpreamble|, executed just after
  \verb|\begin{theindex}| (\verb|\relax| by default).
\item \verb|\indexpostamble|, executed before 
  \verb|\end{theindex}| (\verb|\relax| by default).
\item \verb|\indexskip|, which is \verb|\indexspace| by default, 
with similar purpose.
\item \verb|\indexheading|, is defined as follows:
\begin{verbatim}
\newcommand\indexheading[1]{%
  {\bfseries\hfil
   \MakeUppercase{#1}%
   \hfil}%
  \nopagebreak}
\end{verbatim}
\item \verb|\indexrangesep|, which by default is |-|.
\end{itemize}

Remember to set the style to |gindex| when running \textsf{makeindex}:
\begin{verbatim}
  makeindex -s gindex yourfile
\end{verbatim}

Note also there is no need to run \textsf{makeindex} to fine tune the
format.

\section{Internals}

The |ind| files is generated by |gindex.ist| with generic macros like:
\begin{verbatim}
  \addindexitem{bla1}{1}
    \addindexsubitem{subbla1}{3}
    \addindexsubitem{subbla2}{5}
\end{verbatim}
What \textsf{gindex} does is to convert them to the above format. Some
parameters should be cuntomized in the |ist| file, however. These are:
\begin{verbatim}
headings_flag   -1
actual          '@'
quote           '"'
level           '!'
page_precedence "rnaRA"
keyword         "\\indexentry"
delim_n         ", "
\end{verbatim}
As to |headings_flag|, the recommended way to deal with case is with
|\indexheading|.

\end{document}

MIT License
-----------

Permission is hereby granted, free of charge, to any person obtaining a
copy of this software and associated documentation files (the
"Software"), to deal in the Software without restriction, including
without limitation the rights to use, copy, modify, merge, publish,
distribute, sublicense, and/or sell copies of the Software, and to
permit persons to whom the Software is furnished to do so, subject to
the following conditions:

The above copyright notice and this permission notice shall be included
in all copies or substantial portions of the Software.

THE SOFTWARE IS PROVIDED "AS IS", WITHOUT WARRANTY OF ANY KIND, EXPRESS
OR IMPLIED, INCLUDING BUT NOT LIMITED TO THE WARRANTIES OF
MERCHANTABILITY, FITNESS FOR A PARTICULAR PURPOSE AND NONINFRINGEMENT.
IN NO EVENT SHALL THE AUTHORS OR COPYRIGHT HOLDERS BE LIABLE FOR ANY
CLAIM, DAMAGES OR OTHER LIABILITY, WHETHER IN AN ACTION OF CONTRACT,
TORT OR OTHERWISE, ARISING FROM, OUT OF OR IN CONNECTION WITH THE
SOFTWARE OR THE USE OR OTHER DEALINGS IN THE SOFTWARE.
